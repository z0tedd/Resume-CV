%%%%%%%%%%%%%%%%%%%%%%%%%%%%%%%%%%%%%%%%%
% Developer CV
% LaTeX Template
% Version 1.0 (28/1/19)
%
% This template originates from:
% http://www.LaTeXTemplates.com
%
% Authors:
% Jan Vorisek (jan@vorisek.me)
% Based on a template by Jan Küster (info@jankuester.com)
% Modified for LaTeX Templates by Vel (vel@LaTeXTemplates.com)
% Modified for myself by Anton Faitelson (z0tedd@gmail.com)
% License:
% The MIT License (see included LICENSE file)
%
%%%%%%%%%%%%%%%%%%%%%%%%%%%%%%%%%%%%%%%%%
%RECOMMENDED - fontawesome5
%----------------------------------------------------------------------------------------
%	PACKAGES AND OTHER DOCUMENT CONFIGURATIONS
%----------------------------------------------------------------------------------------

\documentclass[9pt]{developercv} % Default font size, values from 8-12pt are recommended
\usepackage[russian]{babel}
%----------------------------------------------------------------------------------------

\begin{document}

%----------------------------------------------------------------------------------------
%	TITLE AND CONTACT INFORMATION
%----------------------------------------------------------------------------------------

\begin{minipage}[t]{0.45\textwidth} % 45% of the page width for name
	\vspace{-\baselineskip} % Required for vertically aligning minipages
	
	% If your name is very short, use just one of the lines below
	% If your name is very long, reduce the font size or make the minipage wider and reduce the others proportionately
	\colorbox{black}{{\HUGE\textcolor{white}{\textbf{\MakeUppercase{Anton}}}}} % First name
	
	\colorbox{black}{{\HUGE\textcolor{white}{\textbf{\MakeUppercase{Faitelson}}}}} % Last name
	
	\vspace{6pt}
	
  {\huge Developer, DevOps \& a good man} % Career or current job title
\end{minipage}
\begin{minipage}[t]{0.275\textwidth} % 27.5% of the page width for the first row of icons
	\vspace{-\baselineskip} % Required for vertically aligning minipages
	
	% The first parameter is the FontAwesome icon name, the second is the box size and the third is the text
	% Other icons can be found by referring to fontawesome.pdf (supplied with the template) and using the word after \fa in the command for the icon you want
	\icon{MapMarker}{12}{Russia, Kursk}\\
	\icon{Phone}{12}{+89103122108}\\
	\icon{At}{12}{\href{mailto:z0tedd@gmail.com}{z0tedd@gmail.com}}\\	
\end{minipage}
\begin{minipage}[t]{0.275\textwidth} % 27.5% of the page width for the second row of icons
	\vspace{-\baselineskip} % Required for vertically aligning minipages
	
	% The first parameter is the FontAwesome icon name, the second is the box size and the third is the text
	% Other icons can be found by referring to fontawesome.pdf (supplied with the template) and using the word after \fa in the command for the icon you want
	\icon{Globe}{12}{\href{https://TODO:_Add_name_please}{z0tedd.tech}}\\
	\icon{Github}{12}{\href{https://github.com/alyxvance}{github.com/z0tedd}}\\
	\icon{Telegram}{12}{\href{https://t.me/z0tedd}{@z0tedd}}\\
\end{minipage}

\vspace{0.5cm}

%----------------------------------------------------------------------------------------
%	INTRODUCTION, SKILLS AND TECHNOLOGIES
%----------------------------------------------------------------------------------------

\cvsect{Who Am I?}

\begin{minipage}[t]{0.4\textwidth} % 40% of the page width for the introduction text
	\vspace{-\baselineskip} % Required for vertically aligning minipages
  Меня зовут Файтельсон Антон, мне 18 лет, родился и вырос в центральной части города Курск. 
  Я обучаюсь по направлению "МОАИС" в Курском государственном университете. Работал в офисе 
  местной студии по разработки сайтов, долгое время я занимался бекэндом на Python, в данный 
  момент изучаю Data Science. Иногда я пишу книги, скоро будет выпущен мой сборник анекдотов. 
  Я люблю прикольные штуки, красивых девушек, а также группу "Король и Шут". 
\end{minipage}
\hfill % Whitespace between
\begin{minipage}[t]{0.5\textwidth} % 50% of the page for the skills bar chart
	\vspace{-\baselineskip} % Required for vertically aligning minipages
	\begin{barchart}{5.5}

		\baritem{C++}{60}
		\baritem{Python}{70}
		\baritem{LaTeX}{55}
		\baritem{Linux}{40}
		\baritem{Docker}{30}
		\baritem{Git}{50}
    \baritem{Nginx}{20}
    %\baritem{Neovim}{85}

	\end{barchart}
\end{minipage}

\begin{center}
	\bubbles{5/Neovim, 3/bash, 4/Jira, 5/Linux}
\end{center}

%----------------------------------------------------------------------------------------
%	EXPERIENCE
%----------------------------------------------------------------------------------------

\cvsect{Experience}

\begin{entrylist}

	\entry
		{2021 -- 2022\\\footnotesize{part time}}
		{Backend developer}
		{Desmax studio}
		{
      Я подрабатывал в небольшой студии разработки сайтов.
      Мы разработали сайты для местных индивидуальных предпринимателей, 
      а также для малого бизнеса, к сожалению многие наши 
      клиенты уже закрылись.\\
    \texttt{Python}\slashsep\texttt{MariaDB}\slashsep\texttt{Linux}}
\end{entrylist}

%----------------------------------------------------------------------------------------
%----------------------------------------------------------------------------------------
%	COMPETITIONS\COURSES
%----------------------------------------------------------------------------------------


\cvsect{Competitions \textbackslash \space Courses}
\begin{entrylist}

	\entry
  {2024}
    {"IProfi 2024"}
		{MEPHI\textbackslash \space HSE \textbackslash \space MAI }
		{
В финал я попал сразу по трем направлениям: 
биоинженерия, безопасность информационных объектов
и критически важных объектов, а также электроника, радиотехника 
и системы связи. Все они проходили от разных 
университетов, были чрезвычайно сложные, но невероятно интересные.
    }
\end{entrylist}

\begin{entrylist}

	\entry
  {2024}
    {Samsung IT Academy Hack 2024}
		{Kursk State University}
		{
      Я учавствовал в команде "Small data and Anton".
      Хоть мы и не заняли призовое место среди всех площадок, но 
      на региональном уровне мы заняли призовое место.
    }
\end{entrylist}

\begin{entrylist}

	\entry
  {2023}
    {RUCODE}
		{Kursk State University}
		{
      Я учавствовал в Всероссийских соревнованиях по спортивному программированию,
      пусть и не занял призового места, но получил крайне приятный опыт.
    }
\end{entrylist}



%----------------------------------------------------------------------------------------
%	EDUCATION
%----------------------------------------------------------------------------------------

\cvsect{Education}

\begin{entrylist}

	\entry
		{2024 -- 2027}
		{Bachelor's Degree}
		{Kursk State University}
    {
      Прохожу обучение по направленю "Математическое обеспечение и администрирование информационных систем". 
      Я наслаждаюсь обучением, чему свидетельствуют по высокие оценки и достижения.
    }
\end{entrylist}

%----------------------------------------------------------------------------------------
%	ADDITIONAL INFORMATION
%----------------------------------------------------------------------------------------

\begin{minipage}[t]{0.3\textwidth}
	\vspace{-\baselineskip} % Required for vertically aligning minipages

  \cvsect{Languages}\\
	\textbf{Russian} - native\\
	\textbf{English} - proficient\\
	\textbf{Hebrew} - rudimentary\\
	\textbf{Arabic} - rudimentary\\
\end{minipage}
\hfill
\begin{minipage}[t]{0.3\textwidth}
	\vspace{-\baselineskip} % Required for vertically aligning minipages
	
	\cvsect{Hobbies}
	
  Я люблю вкусно поесть, а также вкусно готовить. Но мое главное хобби — чтение книг. 
  Из физической активности я предпочитаю ходить пешком 5 км.
\end{minipage}
\hfill
\begin{minipage}[t]{0.3\textwidth}
	\vspace{-\baselineskip} % Required for vertically aligning minipages
	
	\cvsect{Non profit}

Я помогаю всем своим друзьям с ремонтом компьютеров, а также друзьям-биологам с анализом наборов данных.
\end{minipage}

%----------------------------------------------------------------------------------------

\end{document}
